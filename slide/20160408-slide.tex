\documentclass[12pt, unicode]{beamer}
\usetheme{Warsaw}
\usepackage{luatexja}
\usepackage{listings}
%color
\definecolor{battleshipgrey}{rgb}{0.52, 0.52, 0.51}

\title{Fluentd forward protocol modes in v0}
\author{Hiroshi Hatake}
\date[2016/04/08]{情報共有の会}

\begin{document}

\frame{\maketitle}

\begin{frame}{Fluentd forward protocol modes in v0}
\end{frame}

\begin{frame}{introduction}
\begin{block}{Forward protocol\footnote[frame]{https://github.com/fluent/fluentd/wiki/Forward-Protocol-Specification-v0}}
Fluentd has three modes in forward protocol.
\end{block}
\begin{itemize}
\item<1-> Message Mode
\item<2-> Forward Mode
\item<3-> PackedForward Mode
\end{itemize}
\end{frame}

\begin{frame}{introduction | Message Mode}
\begin{block}{Message Mode\footnote[frame]{https://github.com/fluent/fluentd/wiki/Forward-Protocol-Specification-v0\#message-mode}}
It carries just an event.
\end{block}
\begin{itemize}
\item \uncover<1->{\textbf {tag} is a string separated with '.' (e.g. myapp.access) to categorize events.}
\item \uncover<1->{\textbf {time} is a number of seconds since Unix epoch.}
\item \uncover<1->{\textbf {record} is key-value pairs of the event record.}
\item \uncover<2->{\color{battleshipgrey} \textbf {option} is optional key-value pairs, to bring data to control servers' behavior.}
\end{itemize}
\end{frame}

\begin{frame}{introduction | Message Mode}
\begin{block}{Message Mode\footnote[frame]{https://github.com/fluent/fluentd/wiki/Forward-Protocol-Specification-v0\#message-mode}}
It carries just an event.
\end{block}
\begin{table}[htb]
\resizebox{\textwidth}{!}{%
    %\caption{Message mode protocol}
    \begin{tabular}{l | c | c | c } \hline
      name & type & content & mandatory? \\ \hline \hline
      tag & str & tag name & yes \\ \hline
      time & Integer/EventTime(ext type) & Unix Time/Unix Time with nano seconds & yes \\ \hline
      record & Object(like HashMap) & pairs of keys(String) and values(Object) & yes \\ \hline
      option & Object(like HashMap) & pairs of keys(String) and values(Object) & no \\ \hline
    \end{tabular}}
\end{table}
\end{frame}

\begin{frame}{introduction | Forward Mode}
\begin{block}{Forward Mode\footnote[frame]{https://github.com/fluent/fluentd/wiki/Forward-Protocol-Specification-v0\#forward-mode}}
It carries a series of events as a msgpack array on a single request.
\end{block}
\begin{itemize}
\item \uncover<1->{\textbf {tag} is a string separated with '.' (e.g. myapp.access) to categorize events.}
\item \uncover<1->{\textbf {entries} is an array of Entries which are pairs of Unix epoch and record.}
\item \uncover<2->{\color{battleshipgrey} \textbf {option} is optional key-value pairs, to bring data to control servers' behavior.}
\end{itemize}
\end{frame}

\begin{frame}{introduction | Message Mode}
\begin{block}{Forward Mode\footnote[frame]{https://github.com/fluent/fluentd/wiki/Forward-Protocol-Specification-v0\#forward-mode}}
It carries a series of events as a msgpack array on a single request.
\end{block}
\begin{table}[htb]
\resizebox{\textwidth}{!}{%
    %\caption{Message mode protocol}
    \begin{tabular}{l | c | c | c } \hline
      name & type & content & mandatory? \\ \hline \hline
      tag & str & tag name & yes \\ \hline
      entries & Array & an array of Entries which are pairs of Unix epoch and record. & yes \\ \hline
      option & Object(like HashMap) & pairs of keys(String) and values(Object) & no \\ \hline
    \end{tabular}}
\end{table}
\end{frame}

\begin{frame}{introduction | PackedForward Mode}
\begin{block}{PackedForward Mode\footnote[frame]{https://github.com/fluent/fluentd/wiki/Forward-Protocol-Specification-v0\#packedforward-mode}}
It carries a series of events as a msgpack array on a single request.
\end{block}
\begin{itemize}
\item \uncover<1->{\textbf {tag} is a string separated with '.' (e.g. myapp.access) to categorize events.}
\item \uncover<1->{\textbf {entries} is a msgpack stream of Entry which contains pairs of Unix epoch and record msgpack binary.}
\item \uncover<2->{\color{battleshipgrey} \textbf {option} is optional key-value pairs, to bring data to control servers' behavior.}
\end{itemize}
\end{frame}

\begin{frame}{introduction | Message Mode}
\begin{block}{PackedForward Mode\footnote[frame]{https://github.com/fluent/fluentd/wiki/Forward-Protocol-Specification-v0\#packedforward-mode}}
It carries a series of events as a msgpack array on a single request.
\end{block}
\begin{table}[htb]
\resizebox{\textwidth}{!}{%
    %\caption{Message mode protocol}
    \begin{tabular}{l | c | c | c } \hline
      name & type & content & mandatory? \\ \hline \hline
      tag & str & tag name & yes \\ \hline
      entries & str | bin & a msgpack stream of Entry which contains of Unix epoch and record. & yes \\ \hline
      option & Object(like HashMap) & pairs of keys(String) and values(Object) & no \\ \hline
    \end{tabular}}
\end{table}
\end{frame}

\begin{frame}{How to implement minimal Fluentd logger?}
\begin{block}{Minimal Fluentd logger implementation strategy}
\begin{itemize}
\item \uncover<1-> {{\textbf {Fluentd logger}} needs to send an event to Fluentd.}
  \begin{itemize}
  \item \uncover<2-> {{\textbf {Fluentd logger}} need {\textbf {not}} to add optional elements in sending events to Fluentd.}
  \item \uncover<3-> {{\textbf {Fluentd logger}} {\textbf {should}} use message mode to send events to Fluentd.}
  \end{itemize}
\end{itemize}
\end{block}
\end{frame}

\begin{frame}{How to implement minimal Fluentd logger?}
\begin{block}{Minimal Fluentd logger implementation}
\begin{itemize}
\item \uncover<1-> {{\textbf {Fluentd logger}} needs to connect to Fluentd with TCP.}
\item \uncover<2-> {{\textbf {Fluentd logger}} send an event which conatins tag, timespec, and record, to Fluentd.}
\item \uncover<3-> {{\textbf {Fluentd logger}} should disconnect against  Fluentd.}
\end{itemize}
\end{block}
\end{frame}

%% Define highlighting for Rust.
\lstdefinelanguage{Rust} {
  morecomment = [l]{//},
  morecomment = [l]{///},
  morecomment = [s]{/*}{*/},
  morestring=[b]",
  sensitive = true,
  morekeywords = {extern,  crate,  fn,  let, use}
}
% Set font size for lstlisting.
\newcommand\Small{\fontsize{9}{9.2}\selectfont}
\begin{frame}[fragile]
Fruently example:
\begin{lstlisting}[language={Rust},basicstyle=\ttfamily\Small]
extern crate fruently;
use fruently::fluent::Fluent;
use std::collections::HashMap;
use fruently::forwardable::JsonForwardable;

fn main() {
    // create record
    let mut obj: HashMap<String, String> = HashMap::new();
    obj.insert("name".to_string(), "fruently".to_string());
    // establish connection with Fluentd.
    let fruently = Fluent::new("0.0.0.0:24224", "test");
    // send record to Fluentd and disconnecting automatically.
    let _ = fruently.post(&obj);
}
\end{lstlisting}
\end{frame}

\frame{\centering \Large Thank you for your attention.}

\end{document}
